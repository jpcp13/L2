\documentclass{article}

\usepackage[utf8]{inputenc}
\usepackage{amsthm}
\usepackage{amssymb}
\usepackage{amsfonts}
\usepackage[francais]{babel}
\usepackage{fancyvrb}
\usepackage{hyperref}

\title{TP 1}
\author{LAICHE Issam et BELOUCIF Malik}

\begin{document}
\maketitle

Dans ce tp élaboré par Mr Cardinal, nous débutons python avec des mathématiques.\newline

\begin{enumerate}

\item{\bf Création d'une fonction et observation}
	
On a appris a écrire et définir une fonction pour la première fois en python.

La fonction $f$ est définie par $f(x) = x^2-x-1$


On a aussi appris a afficher une fonction graphiquement
Pour afficher la fonction graphiquement on a du utiliser la fonction matplotlib.pyplot. Qui est super interessante car elle nous permet d'afficher plusieurs caracteristique du graphique (couleur de la courbe cadrillage du graphe... )\newline

On a donc remarqué que la fonction avait 2 racine parmi eux il y a le nombre d'or.\newline


\item{\bf Recherche d’un zéro d’une fonction au moyen de la méthode du point fixe.}

L'ennoncé nous a donné une fonction auxiliaire $g$ définie par $g(x) = 1+1/x$

On a vérifier que les 0 de la fonction $f$ était des point fixe de $g$\newline
Ensuite à l'aide d'une suite définie par : $u_{n+1}=f(u_n)$ on a etudié la convergence de la suite vers le nombre d'or.\newline

\item{\bf Implémentation d’une fonction python point fixe.}

Nous avons ensuite créer une fonction en python qui permet de géneraliser la recherche des points fixes avec une fonction quelquonque. (cf fichier source)\newline Nous avons ensuite fais des test avec la fonction $g$ pour trouver les 2 points fixe trouvé en amont.
\newline Sur papier nous avons dessinés la fonction $g$ et la première bissectrice $f(x)=x$
 et on a remarqué que le point fixe 1.618... etait attractif ce qui explique la convergence ci dessus et inversement pour l'autre point fixe.\newline

\item{\bf Calculer les 0 d'une fonction avec la methode de newton.}

La méthode de newton repose sur le fait d'appliquer la methode du point fixe avec $g(x)=x-\frac{f(x)}{f'(x)}$

Donc comme précedement on a verifier que les racines de $f$ étaient bien des points fixes de $g$ et c'est logique car si ce sont des racines de $f$ alors $\frac{f(x)}{f'(x)}=0$

On a donc ensuite ecris une fonction $newton$ en python qui donne une approximation des racines d'une fonction $f$
Pour ce faire on a eu besoin de Mr cardinal car il y avait un autre argument a mettre qui est $df$ soit la dérivé de $f$.

Enfin on à effectuer des tests (cf fichier source).\newline

\item{\bf Recherche encore des racines mais avec la methode de la séecante.}

L'avantage de cette methode c'est que nous n'avons pas besoin de la derivée.



Cette methode repose sur de la recurence, on approche les racines par rapport a l'approchement fait precedement. Il faut donc partir de quelques choses et pour ce faire il faut partir de l'encadrement donné et calculer les premiers termes de la suite $x_{n+1}=x_n-\frac{x_n-x_{n-1}}{f(x_n)-f(x_{n-1})}f(x_n)$.

On a donc écris une fonction qui permet de calculer l'approximation de 2 valeurs initiales $x_0$ et $x_1$ avec le calcul recurent de la suite $x_{n+1}$ si-dessus.

Enfin nous avons encore effectuer des tests (cf fichier sources).\newline

\item{\bf Dernière methode la methode dite de dichotomie.}

La dichotomie, dans ce cas, se base sur diviser un intervalle en 2 pour le retrecir, et donc ce rapprocher d'un point en rendant sont intervalle infiniment petit.

Dans notre cas comme on cherche un 0 et que la fonction, continue à un certain intervalle aura des nombres negatifs et d'autres positifs, on pourra donc ce rapprocher du 0.

On a donc écris une fonction qui va nous envoyer un encadrement de la racine de la longueur souhaité pour se rapprocher de celles-ci.


Enfin nous avons encore effectuer des tests (cf fichier sources).\newline\newline

conclusion: On a testé plusieurs methodes afin de trouver des racines et chauqe methode a ces avantages et ces incoveniant.





\end{enumerate}
\end{document}

