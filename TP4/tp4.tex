\documentclass{exam}

\usepackage[utf8]{inputenc}
\usepackage{amsmath}
\usepackage{amssymb}
\usepackage{amsthm}
\usepackage{amssymb}
\usepackage{amsfonts}
\usepackage[francais]{babel}
\usepackage{fancyvrb}
\usepackage{hyperref}

\title{TP4\\ Arithmétique}

\begin{document}
\maketitle

\begin{questions}


\question
{\bf Un nombre est-il premier ?}
\begin{parts}
\part
Chercher comment on obtient en python le quotient et le reste de la division euclidienne d'un entier par un autre entier.
\part
Ecrire la définition d'un nombre premier.
\part
Parmi les entiers suivants, dire lesquels sont premiers : $1001, 2017, 3001, 49999, 89999$.
\part
Ecrire une fonction python \texttt{is\_prime} qui prend en argument un entier $n$ et qui renvoie \texttt{true} si $n$ est premier, \texttt{false} sinon.
\part
Les nombres de Fermat $F_n$ sont définis par $F_n = 2^{2^n} + 1$; les nombres $F_0, F_1, \cdots, F_5$ sont ils premiers ?
\end{parts}

\question
{\bf Crible d'Erasthotène, distribution des nombres premiers.}
\begin{parts}
\part
A l'aide du \href{https://en.wikipedia.org/wiki/Sieve_of_Eratosthenes}{crible d'Erathostène}, calculer la liste de tous les nombres premiers inférieurs à $200$.
\part
Ecrire une fonction python \texttt{primes} qui prend en argument un entier $n$ et qui renvoie la liste de tous les premiers inférieurs à $n$.
\part
Calculer la liste de tous les premiers inférieurs à $1000$; écrire cette liste dans un fichier primes.txt, dix nombres par ligne.
\part
On note $\pi(n)$ le nombre d'entiers premiers inférieurs à $n$. Représenter graphiquement $\pi(n)$ en fonction de $n$ pour $n$ variant de $2$ à $1000$. Sur le même graphique, rajouter la fonction $\dfrac{n}{\log n}$. Qu'observe-t'on ? voir le \href{https://en.wikipedia.org/wiki/Prime_number_theorem}{théorème des nombres premiers}.
\part
Créer en python la table suivante, la remplir et l'écrire dans un fichier texte.
\begin{center}
\begin{tabular}{r | c | c}
$n$ & $\pi(n)$ & $\frac{n}{\log n}$ \\
\hline
$10^1$ & {} & {}\\
$10^2$ & {} & {}\\
$10^3$ & {} & {}\\
$10^4$ & {} & {}\\
$10^5$ & {} & {}\\
$10^6$ & {} & {}
\end{tabular}
\end{center}
\end{parts}


\question
{\bf Factorisation d'un entier en premiers.}
\begin{parts}
\part
Ecrire le \href{https://en.wikipedia.org/wiki/Fundamental_theorem_of_arithmetic}{théorème fondamental de l'arithmétique}.
\part
Calculer la décomposition en facteurs premiers de $924$.
\part
Ecrire une fonction python \texttt{factors} qui prend en arguments un entiers $n$ et qui renvoie la liste, dans l'ordre croissant, des facteurs premiers de $n$, chaque facteur étant répété autant de fois que nécessaire. Ainsi, pour $n = 60$, on obtiendra $[2, 2, 3, 5]$.
\end{parts}


\question
{\bf pgcd de deux entiers, identité de Bézout, algorithme d'Euclide.}
\begin{parts}
\part
Ecrire la définiton du pgcd de deux entiers.
\part
En utilisant la décomposition en facteurs premiers, calculer le pgcd des deux nombres $a = 4864, b = 3458$.
\part
Ecrire l'énoncé de l'\href{https://en.wikipedia.org/wiki/B%C3%A9zout%27s_identity}{identité de Bézout}
\part
A l'aide de l'\href{https://en.wikipedia.org/wiki/Extended_Euclidean_algorithm}{algorithme d'Euclide étendu}, calculer $d$, le pgcd des deux nombres $a = 4864, b = 3458$, ainsi que les coefficients de Bézout $x, y$ tels que $xa + yb = d$.
\part
Ecrire une fonction python \texttt{euclide} qui prend en arguments deux entiers $a, b$ et qui renvoie $x, y, d$, où $d$ est le pgcd de $a, b$ et $x, y$ les coefficients de Bézout.
\end{parts}


\question
{\bf Algorithme RSA.}\\
Ecrire les codes python, mais ne pas rédiger de compte rendu sur cette partie.
\begin{parts}
\part
Voir le fonctionnement du \href{https://fr.wikipedia.org/wiki/Chiffrement_RSA}{chiffrement RSA}.
\part
A l'aide de la fonction python \texttt{random.randint()} et de votre fonction \texttt{is\_prime}, choisir deux entiers premiers distincts $p, q$ supérieurs à $2^{35}$ et inférieurs à $2^{36}$.
\part
 On pose $n=pq$; vérifier que $n$ est supérieur à $2^{70}$. On sait bien sûr que $n$ n'est pas premier, mais que se passe-t'il lorsqu'on éxecute \texttt{is\_prime(n)} ? Expliquer le phénomène observé.
\part
Voir la définition de la fonction \href{https://fr.wikipedia.org/wiki/Indicatrice_d%27Euler}{indicatrice d'Euler}.
Que vaut $\phi(n)$ pour l'entier $n$ défini ci-dessus ?
\part
Choisir aléatoirement un entier naturel $e$ premier avec $\phi(n)$, et strictement inférieur à $\phi(n)$; $e$ est appelé exposant de {\bf chiffrement}.
\part
Calculer l'entier naturel $u$ inverse de $e$ modulo $\phi(n)$, et strictement inférieur à $\phi(n)$;  $u$ est appelé clé de {\bf déchiffrement}; indication : si $d$ est le pgcd de $e$ et $\phi(n)$ et si $u, v$ sont les coefficients de Bézout tels que $d = ue + v\phi(n)$, alors $u$ est inverse de $e$ modulo $\phi(n)$; de plus dans l'identité de Bézout on peut toujours choisir $u$ compris entre $1$ et $n$.
\part
Choisir $M$ un entier naturel strictement inférieur à $n$; $M$ représente le message que l'on veut chiffrer. Calculer $C = M^e$ modulo $n$ - indication : il faut soit implémenter la \href{https://fr.wikipedia.org/wiki/Exponentiation_rapide}{méthode d'exponentiation rapide} en effectuant les produits modulo $n$, soit utiliser la fonction python built-in \texttt{pow()}; $C$ est le message chiffré, utilisant la clé publique de chiffrement $e, n$.
\part
 Vérifier que l'on retrouve bien $M$ en effectuant l'opération $C^u$ modulo $n$; le couple $u, n$ est la clé privée de déchiffrement.
\part
En rassemblant les éléments précédents, écrire une fonction python \texttt{rsa\_key()}, qui ne prend pas d'argument, et qui renvoie les entiers $e, u, n$, où $e, n$ servira de clé publique de chiffrement et $u, n$ de clé privée de déchiffrement.
\part
Combien y a t'il de chaines de $10$ caractères ASCII distinctes possibles, sachant qu'un caractère ASCII est codé sur $7$ bits ? En utilisant la fonction built-in \texttt{ord()}, écrire une fonction python \texttt{tenchars2int} qui prend en argument une chaine de caractères \texttt{tenchars} de longueur $10$ au plus et qui renvoie un entier $M$, inférieur strictement à $2^{70}$, codant la chaine de caractères \texttt{tenchars}. De même, en utilisant la fonction built-in \texttt{chr()}, écrire une fonction python \texttt{int2tenchars} qui prend en argument un entier $M$ inférieur strictement à $2^{70}$ et qui renvoie une chaine de $10$ caractères ASCII. Vérifier que les fonctions  \texttt{tenchars2int} et \texttt{int2tenchars} sont bien inverses l'une de l'autre.
\part
En rassemblant les éléments précédents, écrire une fonction python \texttt{chiffrer()} qui prend en arguments une chaine de caractères de longueur arbitraire et qui renvoie cette chaine chiffrée au moyen d'une clé publique de chiffrement générée par la fonction \texttt{rsa\_key()} - indication : découper le texte à chiffrer en segments de longueur $10$. De même, écrire une fonction python \texttt{dechiffrer()} qui prend en arguments un texte chiffré, ainsi que la clé privée de déchiffrement associée à la clé publique, et qui renvoie le texte déchiffré. Tester les fonctions \texttt{chiffrer()} et \texttt{dechiffrer()} sur des exemples.
\end{parts}

\end{questions}

\end{document}

%%% Local Variables:
%%% mode: latex
%%% TeX-master: t
%%% End:
