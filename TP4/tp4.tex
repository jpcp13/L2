\documentclass{exam}

\usepackage[utf8]{inputenc}
\usepackage{amsmath}
\usepackage{amssymb}
\usepackage{amsthm}
\usepackage{amssymb}
\usepackage{amsfonts}
\usepackage[francais]{babel}
\usepackage{fancyvrb}
\usepackage{hyperref}

\title{TP4\\ Arithmétique}

\begin{document}
\maketitle

\begin{questions}


\question
{\bf Un nombre est-il premier ?}
\begin{parts}
\part
Chercher comment on obtient en python le quotient et le reste de la division euclidienne d'un entier par un autre entier. Rappeler la définition d'un nombre premier. Parmi les entiers suivants, dire lesquels sont premiers : $1001, 2017, 3001, 49999, 89999$.
\part
Ecrire une fonction python \texttt{is\_prime} qui prend en argument un entier $n$ et qui renvoie \texttt{true} si $n$ est premier, \texttt{false} sinon.
\part
Les nombres de Fermat $F_n$ sont définis par $F_n = 2^{2^n} + 1$; les nombres $F_0, F_1, \cdots, F_5$ sont ils premiers ?
\end{parts}

\question
{\bf Crible d'Erasthotène; distribution des nombres premiers.}
\begin{parts}
\part
A l'aide du crible d'Erathostène \url{https://en.wikipedia.org/wiki/Sieve_of_Eratosthenes}, calculer la liste de tous les nombres premiers inférieurs à $200$.
\part
Ecrire une fonction python \texttt{primes} qui prend en argument un entier $n$ et qui renvoie la liste de tous les premiers inférieurs à $n$.
\part
Calculer la liste de tous les premiers inférieurs à $1000$; écrire cette liste dans un fichier primes.txt, dix nombres par ligne.
\part
On note $\pi(n)$ le nombre d'entiers premiers inférieurs à $n$. Représenter graphiquement $\pi(n)$ en fonction de $n$ pour $n$ variant de $2$ à $1000$. Sur le même graphique, rajouter la fonction $\dfrac{n}{\log n}$. Qu'observe-t'on ? voir le Théorème des nombres premiers \url{https://en.wikipedia.org/wiki/Prime_number_theorem}.
\part
Créer en python la table suivante, la remplir et l'écrire dans un fichier texte.
\begin{center}
\begin{tabular}{r | c | c}
$n$ & $\pi(n)$ & $\frac{n}{\log n}$ \\
\hline
$10$ & {} & {}\\
$100$ & {} & {}\\
$1000$ & {} & {}\\
$10000$ & {} & {}\\
$100000$ & {} & {}\\
$1000000$ & {} & {}
\end{tabular}
\end{center}
\end{parts}


\question
{\bf Factorisation d'un entier en premiers.}
\begin{parts}
\part

\end{parts}


\question
{\bf pgcd de deux entiers, identité de Bézout, algorithme d'Euclide.}
\begin{parts}
\part
Rappeler la définiton du pgcd de deux entiers relatifs ainsi que l'énoncé de l'identité de Bézout.
\part
A l'aide de l'algorithme d'Euclide étendu \url{https://en.wikipedia.org/wiki/Extended_Euclidean_algorithm}, calculer le pgcd $d$ des deux nombres $a = 4864, b = 3458$ ainsi que les coefficients de Bézout $x, y$ tels que $xa + yb = d$.
\part
Ecrire une fonction python \texttt{euclide} qui prend en arguments deux entiers $a, b$ et qui renvoie $x, y, d$, où $d$ est le pgcd de $a, b$ et $x, y$ les coefficients de Bézout.
\end{parts}




\end{questions}

\end{document}

%%% Local Variables:
%%% mode: latex
%%% TeX-master: t
%%% End:
